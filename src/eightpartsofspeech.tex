\section{Eight Parts of Speech}

\subsection{The Meaning and Functions of the Eight Parts of Speech}

\subsubsection{Nouns}
Meaning: words used to name persons or things.
\newline
\newline

Examples: Susan, China, bird, crowd, love, milk
\begin{itemize}
    \item The \underline{boy} is my \underline{brother}.
    \item \underline{Jack} lives in \underline{Canada}.
    \item \underline{Failure} is the \underline{mother} of \underline{success}.
\end{itemize}

Functions: as subject, object, complement, apposition, object of a preposition
\begin{itemize}
    \item \underline{Edison} is my best friend. (Subject)
    \item We like Edison. (Object)
    \item He is Edison. (Complement)
    \item My best friend, Edison, is coming. (Apposition)
    \item I am going swimming with Edison. (Object of a preposition)
\end{itemize}

\subsubsection{Pronoun}
Meaning: words used to replace nouns.
\newline
\newline
Examples: I, you, he, she, this, that, them
\begin{itemize}
    \item Your bicycle is better than \underline{mine}.
    \item \underline{You} and \underline{he} came here a week before
        \underline{me}
\end{itemize}

Function: to replace nouns
\begin{itemize}
    \item Stanley is a Chinese boy. \underline{He} (= Stantley) is wise.
    \item I have a watch and my brother has also \underline{one} (= a watch).
        \underline{Mine} (= My watch) keeps very good time but \underline{his}
        (= his watch) doesn't.
\end{itemize}

\subsubsection{Adjectives}
Meaning: words used to define or limit nouns.
\newline
\newline
Examples: beautiful, little, this, a, an, the, one, two
\begin{itemize}
    \item Coningham is \underline{an} \underline{honest} student.
    \item I have \underline{two} brothers and \underline{three} sisters.
    \item Sunday is \underline{the} \underline{first} day of \underline{the}
        week.
\end{itemize}

Function: to modify nouns or pronouns
\begin{itemize}
    \item Catherine is an \underline{American} girl. She is \underline{tall}.
    \item \underline{Which} book will you read? I will read \underline{this}
        \underline{red} one.
\end{itemize}

\subsubsection{Verbs}
Meaning: words used to indicate what somebody or something does
\newline
\newline
Examples: is, am, are, come, speak, have
\begin{itemize}
    \item Elizabeth \underline{was} a great dances. She \underline{won} many
        prizes.
    \item The ship \underline{left} Hongkong to London.
    \item I \underline{am} \underline{going} to the school.
\end{itemize}

Function: words used for saying something about a person or thing.
\begin{itemize}
    \item She \underline{swims} very well.
    \item There \underline{are} two girls in the room.
    \item She \underline{is} fond of music.
\end{itemize}

\subsubsection{Adverbs}
Meaning: words used to modify any kind of word except a noun or pronoun.
\newline
\newline
Examples: now, always, suddenly, slowly, yesterday
\begin{itemize}
    \item Johnson spoke \underline{fast}.
    \item Collin \underline{always} stays at home.
    \item They were \underline{very} kind to me.
\end{itemize}

Function:
\begin{itemize}
    \item {\it to modify verbs}
        \begin{itemize}
            \item The soldiers fought \underline{bravely}.
            \item Porter is coming \underline{now}.
            \item Have you \underline{ever} seen a real lion?
        \end{itemize}
    \item {\it to modify adjectives}
        \begin{itemize}
            \item She is \underline{sometimes} careless.
            \item I am \underline{terribly} sorry to hear that.
            \item William is \underline{very} handsome.
        \end{itemize}
    \item {\it to modify adverbs}
        \begin{itemize}
            \item Tom behaves \underline{very} badly.
            \item We go there \underline{quite} often.
        \end{itemize}
\end{itemize}

\subsubsection{Prepositions}
Meaning: words used for showing what one person or thing has to do with another
person of thing
\newline
\newline
Examples: at, on, for, in, with
\begin{itemize}
    \item I look \underline{at} you.
    \item The flowers \underline{in} the vase are roses.
    \item Thomas insisted \underline{on} studying English.
\end{itemize}

Function: words governig noun or pronoun, expressing latter's relation to
another word
\begin{itemize}
    \item Thomas is anxious \underline{about} his mother's health.
    \item I must apologize \underline{to} seeing you again.
    \item They are looking forward \underline{to} seeing you again.
\end{itemize}

\subsubsection{Conjunctions}
Meaning: words used for joining one word to another word, one phrase to another
prhase and one clause to another clause
\newline
\newline
Examples: and, but, as, who, which, because, as soon as, although
\begin{itemize}
    \item London \underline{and} New York are large cities.
    \item \underline{Because} I am busy, I cannot go with you.
    \item Joe said \underline{that} he liked it.
    \item My uncle, \underline{who} is a teacher, is coming.
\end{itemize}

Function: words joining words, phrases and clauses
\begin{itemize}
    \item Peter always write quickly \underline{and} carelessly.
    \item You can put the books on the table \underline{or} on the chair.
    \item David works hard \underline{in order that} he should pass the exam.
\end{itemize}

\subsubsection{Interjections}
Meaning: words used as an exclamation
\newline
\newline
Examples: Ah, Oh, Hurrah, Alas, Good Heavens

\subsection{The Same Word in Different Parts of Speech}
It is important to remember that it is the \underline{function} of a word in a
sentence which determines what part of speech a word is.
\newline

{\it A word may belong to two or more part of speech with changing its form:}
\newline

\begin{tabular}{ll}
    1. \textbf{stay} & George \underline{stays} in bed. (verb) \\
    & Cornell made a long \underline{stay} in Beijing. (noun) \\ \\
    2. \textbf{love} & We all \underline{love} our country. (verb) \\
    & Harrison has a \underline{love} of books. (noun) \\ \\
    3. \textbf{walk} & I am used to take a \underline{walk} in the morning. (noun) \\
    & I \underline{walk} along the seashore. (verb) \\ \\
    4. \textbf{fast} & Cooper is a \underline{fast} runner. (adjective) \\
    & Cooper runs \underline{fast}. (adverb) \\
    & After 30 hours Cooper broke his \underline{fast}. (noun) \\
    & Cooper will \underline{fast} tomorrow. (verb).
\end{tabular}
\newline

{\it However, some words' form have to be changed to form different parts of
speech.}
\newline

\begin{tabular}{ll}
    1. \textbf{easy} & The work is \underline{easy}. (adjective) \\
    & They finished their work \underline{easily}. (adverb) \\ \\
    2. \textbf{danger} & The child's life is in \underline{danger}. (noun) \\
    & It is \underline{dangerous} to swim here. (adjective) \\ \\
    3. \textbf{honest} & Richard is an \underline{honest} person. (adjective) \\
    & \underline{Honesty} is the best policy. (noun) \\ \\
    4. \textbf{success} & Failure is the mother of \underline{success}. (noun) \\
    & The performance is \underline{successful}. (adjective) \\
    & We \underline{succeeded} in passing the exam. (verb) \\
    & We have solved this problem \underline{successfully}. (adverb)
\end{tabular}
