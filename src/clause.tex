\section{Clause: Noun Clause}

\subsection{Meaning}

A \underline{Noun Clause} is a clause which does the work of a noun in a
sentence.  It may therefore be the subject, the object, the object to a
preposition and non-finite verbs, complment and apposition.
\newline
\newline

\fbox{
    \begin{tabular}{l|l|l||l}
        & noun clause & & function \\ \hline \hline & & & \\
        & \underline{That my father will come home tomorrow}
        & seems likely.
        & subject \\ & & & \\
        They guessed
        & \underline{what she meant}.
        &
        & object \\ & & & \\
        We laughed at
        & \underline{what he said}.
        &
        & object to a preposition \\ & & & \\
        My wish is
        & \underline{that I can improve my English}.
        &
        & complement \\ & & & \\
        The news
        & \underline{that he is bankrupt}
        & is untrue.
        & apposition \\ & & & \\
        Peter asked her to read
        & \underline{what he had written}.
        &
        & object to an infinitive \\ & & & \\
        Knowing
        & \underline{that she is here}
        & is a comfort to me.
        & object to a gerund \\ & & & \\
        Hearing
        & \underline{what he said},
        & she grew angrily.
        & object to a participle
    \end{tabular}
}
\newline

Note: Naun Clauses are introduced by the following subordinate conjunctions:
\begin{enumerate}
    \item Relative Pronouns: that, who, whoever, whom, which, whichever, what,
        whatever
        \newline
        \begin{tabular}{ll}
            Examples:
            & \underline{Who will help us} is unknown yet. \\ \\
            & \underline{Whoever did that} was a mistake. \\ \\
            & \underline{Whom he wanted to see} puzzled me. \\ \\
            & I do not know \underline{which she has chosen}. \\ \\
            & I believe \underline{that she is sensitive to criticism}. \\ \\
            & That is \underline{what we want to know}.
        \end{tabular}
    \item Relative Adverbs: when, whenever, where, wherever, how, why
        \newline
        \begin{tabular}{ll}
            Examples:
            & \underline{When the meeting will be held} has not yet been
            announced. \\ \\
            & It is remarkable \underline{how he always gets out of trouble}.
            \\ \\
            & This is \underline{where} I found it.
        \end{tabular}
    \item Subordinate Conjunctions: whether, whether ... or, if
        \newline
        \begin{tabular}{ll}
            Examples:
            & \underline{Whether he will come or not} is a problem. \\ \\
            & She was worried about \underline{whether she passed the English
            examination.}. \\ \\
            & He asked me \underline{if I could come}.
        \end{tabular}
\end{enumerate}

\subsection{Remarks in Using Noun Clauses:}
\begin{enumerate}
    \item
        {\it
        As in other types of clauses, the subject always come before the verb.
        }
        \newline
        \newline
        Wrong: Could you tell me what time \st{is it} now?
        \newline
        Right: Could you tell me what time \underline{it is} now?
    \item
        {\it
        In an Adverb Clauses of time, we must use the present simple tense to
        refer to the future.
        }
        \newline
        \newline
        Wrong: I will give him your message when he \st{will come} back
        tomorrow.
        \newline
        Right: I will give him your message when he comes back tomorrow.
        \newline
        \newline
        However, in Noun Clause, we should use the simple future tense when
        referring to the future.
        \newline
        Wrong: She has just told me she \st{comes} earlier tomorrow.
        \newline
        Right: She has just told me she \underline{will come} earlier tomorrow.
\end{enumerate}

\subsection{Use of Noun Clauses}

\begin{enumerate}
    \item Subject
        \newline
        \begin{tabular}{ll}
            Example:
            & \underline{That you will succeed} is certain.
            \\ \\
            & \underline{Who saved the child} remained unknown.
            \\ \\
            & \underline{What we learn in youth} can never be forgotten.
            \\ \\
            & It was not clear \underline{who gave the order}.
            \\ \\
            & It is quite clear \underline{that the crime was done
            deliberately}.
        \end{tabular}
    \item Object
        \newline
        \begin{tabular}{ll}
            Example:
            & I can't see very clearly \underline{who is sitting over there}.
            \\ \\
            & I will show you \underline{whom you are going to meet}.
            \\ \\
            & Please tell me \underline{where the suitcase is}.
            \\ \\
            & We thought it wrong \underline{that she did it}.
            \\ \\
            & I think it likely \underline{that he has failed}.
        \end{tabular}
    \item Object of a Preposition
        \newline
        \begin{tabular}{ll}
            Example:
            & The student always pays attention to \underline{whatever the
            teacher is saying}.
            \\ \\
            & He gets furious against \underline{whoever opposes him}.
            \\ \\
            & The men were paid according to \underline{how much work they did}.
        \end{tabular}
    \item Complement
        \newline
        \begin{tabular}{ll}
            Example:
            & The fact is \underline{that he does not work hard}.
            \\ \\
            & My opinion is \underline{that this story is false}.
            \\ \\
            & The important thing is \underline{what a man does, not what he
            says}.
            \\ \\
            & His problem was \underline{how he should mention the matter}.
        \end{tabular}
    \item Apposition
        \newline
        \begin{tabular}{ll}
            Example:
            & The fact \underline{that he was guity} was plain to everyone.
            \\ \\
            & The news \underline{that we are having a holiday tomorrow} is not
            true.
            \\ \\
            & The theory \underline{that the earth is round} has been proved to
            be true.
        \end{tabular}
    \item Object of a non-finite verb
        \newline
        \begin{tabular}{ll}
            Example:
            & \underline{To understand what this means}, let us study the
            following problem. (infinitive)
            \\ \\
            & \underline{Knowing what a mistake had been made}, he yielded.
            (participle)
            \\ \\
            & Sometimes we are interested in knowing \underline{how fast a given
            piece of work can be done by a machine}. (gerund)
        \end{tabular}
\end{enumerate}

\subsection{Sentence Patterns of Noun Clauses}

\begin{enumerate}
    \item \fbox{It + be + adjective + that + clause}
        \newline
        \newline
        It is certain that William will do well in his exam.
        \newline
        It is strange that she trusted him.
        \newline
        It is important that you study hard.
    \item \fbox{It + be + noun / noun phrase + that + clause}
        \newline
        \newline
        It is common knowledge that the whale is not a fish.
        \newline
        It is a pity that we cannot go.
        \newline
        It is odds that he will come.
    \item \fbox{It + be + part participle + that + clause}
        \newline
        \newline
        It is generally believed that Tom is the best student in the class.
        \newline
        It is demanded that he should leave at once.
        \newline
        It is hoped that Bob will succeed.
\end{enumerate}

\subsection{Noun Phrase $\leftrightarrow$ Noun Clause}
\fbox{
    \begin{tabular}{llll}
        & NP & $\leftrightarrow$ & NC \\ \hline \hline \\
        He denied
        & knowing anything about the conspiracy.
        & $\leftrightarrow$ & that he knew anything about the conspiracy. \\ \\
        Her worry is
        & her son's possible failure in the exam.
        & $\leftrightarrow$ & that her son may fail in the exam. \\ \\
        I insisted
        & on their reviewing the lessons.
        & $\leftrightarrow$ & that they should review the lessons. \\ \\
        I am sure
        & of his honesty.
        & $\leftrightarrow$ & that he is honest. \\ \\
        I don't know
        & the author of this book.
        & $\leftrightarrow$ & who wrote this book. \\ \\
        My pride is wounded by
        & her words.
        & $\leftrightarrow$ & what she said.
    \end{tabular}
}
\newline
\newline

\fbox{
    \begin{tabular}{llll}
        & NP & $\leftrightarrow$ & NC \\ \hline \hline \\
        I doubt
        & the truth of the reportt
        & $\leftrightarrow$ & whether the report is true. \\ \\
        I am cerrtain
        & of giving you satisfaction.
        & $\leftrightarrow$ & that I shall give you satisfaction. \\ \\
        We are glad
        & of your success.
        & $\leftrightarrow$ & that you have succeeded. \\ \\
        To be cast in lead roles is
        & the greatest ambition of the new actress.
        & $\leftrightarrow$ & what the new actress desires most.
    \end{tabular}
}
