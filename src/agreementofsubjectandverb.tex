\section{Agreement of Subject and Verb}

\subsection{General Statement}

\begin{enumerate}
    \item
        {\it
        Subject and verb agree in English
        }
        \newline
        \newline
        \begin{tabular}{llll}
            I am tired. & I was cold. & He has the keys. & Does he look ill? \\
            He is tired. & He was cold. & I have the keys. & Do I look ill? \\
            We are tired. & We were cold. & We have the keys. & Do they look ill?
        \end{tabular}
    \item
        {\it
        Demonstrative adjectives agree with the noun they qualify.
        }
        \newline
        \newline
        \begin{tabular}{ll}
            This book & That boy \\
            These books & Those boys
        \end{tabular}
    \item
        {\it
        The verb after a possessive pronoun depends on what the pronoun refers to.
        }
        \newline
        \newline
        \begin{tabular}{ll}
            This is your shirt. & \underline{Mine IS} over there. (=my shirt is) \\
            These are your shoes. & \underline{Mine ARE} over there. (=my shoes are)
        \end{tabular}
    \item
        {\it
        Verbs like DID in the past tense and also CAN, WOULD, MAY, MIGHT, MUST.
        }
        \newline
        \newline
        \fbox{
            \begin{tabular}{ll}
                You/She/We & \underline{must} hurry.
            \end{tabular}
        }
\end{enumerate}

\subsection{Singular Verbs}
\begin{enumerate}
    \item
        {\it
        ANYBODY, ANYONE, EVERYONE, NOBODY, NO ONE, SOMEBODY, SOMEONE
        }
        \newline
        \newline
        \underline{No one} \textbf{wants} a coca cola.
        \newline
        \textbf{Does} \underline{anybody/anyone} want a coca cola?
        \newline
        \newline
        Mary, \underline{someone/somebody} \textbf{is} waiting for you outside.
        \newline
        \newline
        \underline{No one} \textbf{knows} where the destination is.
    \item
        \fbox{
            \begin{tabular}{llll}
                \underline{One of} & \underline{Every one of} & \underline{Each of} & \multirow{2}{*}{+ plural noun} \\
                \underline{Either of} & \underline{Neither of} & \underline{The number of} &
            \end{tabular}
        }
        \newline
        \newline
        \underline{One of the shops} \textbf{is} still open.
        \newline
        \newline
        \underline{Every one of us} \textbf{is} going to blame him.
        \newline
        \newline
        \underline{Each of the boys} \textbf{has} sung a song to the teacher.
        \newline
        \newline
        \underline{Either of the gentlemen} \textbf{is} good enough to be our teacher.
        \newline
        \newline
        \underline{The number of buses} in Hong Kong \textbf{is} great.
        \newline
        \newline
        He \textbf{is \underline{one of {\it those}}
        $\underbrace{\text{who \underline{{\it were}} sentenced to death.}}_
        {\text{noun clause describing ``those (people)"}}$}
    \item
        \fbox{
            \begin{tabular}{ll}
                Every/Each/Many a & + singular noun
            \end{tabular}
        }
        \newline
        \newline
        \underline{Many a man} \textbf{has} come across ups and downs in life.
        \newline
        \newline
        \underline{Each prisoner} \textbf{is} provided with three meals a day.
        \newline
        \newline
        \textbf{Note: A singular verb is still used when 2 nouns qualified by
        ``each" or ``every" even though they are joined by ``and"}
        \newline
        \newline
        \underline{Every boy and every girl} \textbf{was} at play.
        \newline
        \newline
        \underline{Each man and each woman} \textbf{is} talking.
    \item
        {\it
        A plural noun in fact denotes a single thing.
        }
        \begin{enumerate}
            \item
                {\it
                Name of a book, a country, a house, a hotel, etc.
                }
                \newline
                \newline
                \underline{``Gulliver's Travels"} \textbf{is} still widely read.
                \newline
                \newline
                \underline{``Romeo and Juliet"} \textbf{is} one of the works of William Shakespeare.
                \newline
                \newline
                \underline{The United States} \textbf{is} a highly industrialized country.
            \item
                {\it
                When distances, weights, heights, time, amount of money, etc.
                represent a single figure of quantity.
                }
                \newline
                \newline
                \underline{Ten miles} \textbf{is} quite a long way to walk.
                \newline
                \newline
                \underline{Three pounds of grapes} \textbf{costs} fiften dollars.
                \newline
                \newline
                \underline{Six feet} \textbf{is} too high for me to jump over.
                \newline
                \newline
                \underline{Five years in adversity} \textbf{was} not easy to endure.
                \newline
                \newline
                \underline{Two times two} \textbf{is} equal to four.
            \item
                {\it
                When 2 singular nouns joined by ``and" suggests one single idea
                or refers to the same person / thing.
                }
                \newline
                \newline
                \underline{Bread and butter} \textbf{is} my favourite food.
                \newline
                \underline{Bread and butter} \textbf{are} the necessaries of
                life.
                \newline
                \newline
                {\it
                However when these 2 nouns refer to two persons/things, each
                noun should be preceded by an article respectively. Plural verb
                is used in such a case.
                }
                \newline
                \newline
                \underline{A poet and novelist} \textbf{has} arrived our school.
                \newline
                \underline{A poet and a novelist} \textbf{have} arrived our
                school.
                \newline
                \newline
                \underline{A white and black dog} \textbf{is} lying by the gate.
                \newline
                \underline{A whie and a black dog} \textbf{are} lying by the
                gate.
                \newline
                \newline
                \underline{My son and student} \textbf{does} well in the examination.
                \newline
                \underline{My son and my student} \textbf{do} well in the
                examination.
            \item
                {\it
                Nouns that are plural in form, but singular in meaning.
                }
                \newline
                \begin{tabular}{lllll}
                    apparatus & billiards & crisis & crossroads & measles \\
                    numps & news & summons & taps & whereabouts
                \end{tabular}
                %\newline
                %\newline
                %No news is good news.
                \newline
                \newline
                \underline{His whereabouts} is still unknown.
        \end{enumerate}
    \item
        {\it
        The subject is an uncountable noun.
        }
        \newline
        \newline
        \underline{Rice} \textbf{is} the daily diet of the Chinese.
        \newline
        \newline
        \underline{Milk} \textbf{is} a highly perishable food.
    \item
        {\it ``Class" nouns like `clothing', `food', `furniture', `crockery',
        `cutlery', `luggage', `footwear', `traffic', `stationary', etc.
        }
        \newline
        \newline
        \underline{Fashionable clothing} \textbf{is} always expensive.
        \newline
        \newline
        \underline{The furniture of the sitting-room} \textbf{is} nice.
        \newline
        \newline
        \underline{The luggage} \textbf{is} not heavy.
    \item
        {\it
        When the subject is noun phrase or a noun clause, the verb is usually
        singular.
        }
        \newline
        \newline
        \underline{To increase licence fees and duty on new cars} \textbf{is}
        the government's attempt to control the growth of cars.
        \newline
        \newline
        \underline{Collecting stamps} \textbf{is} my favourite pastime.
        \newline
        \newline
        \underline{That the company has suffered immense loses} \textbf{is} an
        undeniable fact.
\end{enumerate}

\subsection{Plural Verbs}
\begin{enumerate}
    \item
        {\it
        Two or more singular nouns referring to different things are joined by
        ``and".
        }
        \newline
        \newline
        Iron and steel \textbf{are} the most important sources of raw materials
        for heavy industry.
        \newline
        \newline
        Peter and Mary \textbf{are} going to get married.
    \item
        {\it
        FEW, MANY, BOTH, SEVERAL, VARIOUS and the cardinal numbers of TWO,
        THREE, etc. are used with plural verbs.
        }
        \newline
        \newline
        \underline{Few men} \textbf{agree} that women are superior drivers.
        \newline
        \newline
        \underline{Many people} \textbf{prefer} white-collar jobs.
        \newline
        \newline
        \underline{Two buses} \textbf{were} at the station just now.
        \newline
        \newline
        \underline{Both men} \textbf{are wearing} hats.
        \newline
        \newline
        \underline{Several members} \textbf{have disagreed} on the decision made
        by the commitee.
        \newline
        \newline
        \underline{Various seashells} \textbf{are found} on the beach.
    \item
        {\it
        Nouns which are singular in form but plural in meaning.
        }
        \newline
        \newline
        \begin{tabular}{lllll}
            Clergy & Dozen & Gentry & Majority & Cattle \\
            Public & Poultry & People & Police
        \end{tabular}
        \newline
        \newline
        \underline{A dozen eggs} \textbf{have} been cracked.
        \newline
        \newline
        We shall wait until \underline{the police} \textbf{come}.
        \newline
        \newline
        \underline{The majority of the girls} \textbf{like} this kind of sports
        shoes.
        \newline
        \newline
        \underline{The public} \textbf{have} been warned not to throw litter
        everywhere.
    \item
        {\it
        Nouns like
        }
        \begin{enumerate}
            \item COMICS, COSMETICS
                \newline
                \newline
                Nowadays, \underline{cosmetics} \textbf{are} commonly used by
                quite young ladies.
                \newline
                \newline
                \underline{Love comics} \textbf{are} welcomed by teenagers.
            \item AIMS, BARRACKS, BELONGINGS, BOWLS, CARDS, CLOTHES, CONTENTS,
                DAMAGES, GLASSES, GOODS, HEADQUARTERS, MANNERS, PANTS, PLIERS,
                PAJAMAS, REGARDS, RESPECTS, RICHES, SAVINGS, SCISSORS,
                SPECTACLES, SURROUNDINGS, THANKS, TIDINGS, TRAVELS, WAGES,
                WINNINGS.
                \newline
                \newline
                Suck \underline{belongings} of his \textbf{are} worthless.
                \newline
                \newline
                Pleasant \underline{surroundings} \textbf{are} the first
                consideration when choosing a house.
                \newline
                \newline
                It is his \underline{manners} that \textbf{leave} much
                criticism.
        \end{enumerate}
    \item
        {\it
        The subject is a countable noun in plural number.
        }
        \newline
        \newline
        \underline{All the oranges} \textbf{are} sour.
        \newline
        \newline
        \underline{The chairs} \textbf{have} already been removed.
    \item
        {If the subject is a noun formed by putting ``the" before an adjective
        in positive degree, a plural verb is used because it refers to people or
        things of the same kind.
        }
        \newline
        \newline
        For example, ``the rich", ``the poor", ``the strong", ``the weak", etc.
        \newline
        \newline
        In this world, \underline{the strong} \textbf{survive} and
        \underline{the weak} \textbf{perish}.
        \newline
        \newline
        In Hong Kong, \underline{the rich} easily \textbf{become} richer and
        \underline{the poor} \textbf{become} poorer.
\end{enumerate}

\subsection{Verbs Agree with The 1$^{st}$ Subject}
If two subject of different persons and numbers are joined by the below words or
phrases, the verb should agree with the \textbf{FIRST} subject.
\newline
\newline
{\centering
\fbox{
    \begin{tabular}{llllll}
        Together with & With & Along with & Including & As well as & In addition
        to \\ \\
        Accompanied by & Like & Unlike & After & Rather than & And not
    \end{tabular}
}
}
\newline
\newline
The driver as well as two of the passengers was seriously injured in the
accident yesterday.
\newline
\newline
He together with his sons is at work.
\newline
\newline
The captain with his men was saved.
\newline
\newline
The singers, accompanied by the conductor, are ascending the stage.
\newline
\newline
Patrick and not Joseph, has got the job.

\subsection{Verbs Agree with the Last Subject}
If however, the two subjects are connected by the below words and phrases, the
verb should agree with the \textbf{SECOND} subject.
\newline
\newline
{\centering
\fbox{
    \begin{tabular}{lllll}
        Or & Either...Or & Nor & Neither...Nor & Not Only...But Also
    \end{tabular}
}
}
\newline
\newline
Not only the driver but also two of the passengers were seriously injured in the
accident yesterday.
\newline
\newline
He or his neighbour have planted the tree.
\newline
\newline
My brother nor I am going.
\newline
\newline
Neither he nor I am responsible for it.
\newline
\newline
Neither you or the children are coming.

\subsection{Singular or Plural Verbs}
\begin{enumerate}
    \item
        {\it
        The following collective nouns may take either a singular or a plural
        vewrb.
        }
        \newline
        \newline
        {\centering
        \fbox{
            \begin{tabular}{llllll}
                Class & Family & Crew & Staff & Crowd & Council \\ \\
                Team & Audience & Jury & Gang & Government & Committee
            \end{tabular}
        }
        }
        \newline
        \newline
        \begin{enumerate}
            \item
                {\it
                We use a singular verb when the collection is thought of as a
                single unit
                }
                \newline
                \newline
                The class is a big one.
                \newline
                \newline
                There was a large crowd at the scene of the accident.
                \newline
                \newline
                His family is not very large.
                \newline
                \newline
                The committee consists of nine members.
            \item
                {\it
                We use a plural verb when the members of the unit are thought of
                separately.
                }
                \newline
                \newline
                The class are changing their PE uniforms.
                \newline
                \newline
                The crowd are shouting hysterically.
                \newline
                \newline
                The family are all TV fans.
                \newline
                \newline
                The committee are divided in their opinions.
        \end{enumerate}
    \item
        {\it
        The following nouns are followed by either a singular or plural verb
        depending ob the number implied.
        }
        \newline
        \newline
        {\centering
        \fbox{
            \begin{tabular}{llllllll}
                Cattie & Sheep & Deer & Means & Series & Swine & Species &
                Aircraft
            \end{tabular}
        }
        }
        \newline
        \newline
        A means to solve our traffic problems has to be found.
        \newline
        \newline
        Different means to solve our traffic problem have to be found.
        \newline
        \newline
        This series of books provides a five years' course for secondary school.
        \newline
        \newline
        Series of books have been published.
    \item
        {\it
        The following nouns are followed by either a singular or plural verb
        depending on the number implied.
        }
        \newline
        \newline
        {\centering
        \fbox{
            \begin{tabular}{llllllll}
                Athletic & Civics & Economics & Ethics & Gymnastics &
                Linguistics \\
                Mechanics & Physics & Politics & Tactics & Mathematics
            \end{tabular}
        }
        }
        \newline
        \newline
        Economics is the science of the production and distribution of goods.
        \newline
        \newline
        Mathematics is the best subject.
        \newline
        \newline
        The economics of the country are deteriorating.
        \newline
        \newline
        His mathematics are strong.
    \item
        {\it
        We use either a singular or a plural verb after the following words or
        phrases depending on the number of the noun after OF.
        }
        \begin{enumerate}
            \item
                \fbox{
                    \begin{tabular}{llllll}
                        A lot of & A great deal of & Plenty of & &
                        \multirow{2}{*}{+ uncountable noun} &
                        \multirow{2}{*}{$\rightarrow$ a singular verb} \\
                        Most of & Much of & Some of & Half of
                    \end{tabular}
                }
                \newline
                \newline
                \underline{Some of the wood} \textbf{has} been eaten perhaps by mice.
                \newline
                \newline
                \underline{A lot of water} \textbf{has} been drained off.
                \newline
                \newline
                \underline{Half of the land} \textbf{has} been sold.
                \newline
                \newline
                \underline{Most of the rice} \textbf{is imported} from China.
            \item
                \fbox{
                    \begin{tabular}{lllll}
                        A lot of & A great deal of & Plenty of &
                        \multirow{2}{*}{+ countable noun in the plural} &
                        \multirow{2}{*}{$\rightarrow$ a plural verb} \\
                        Most of & Many of & Half of
                    \end{tabular}
                }
                \newline
                \newline
                \underline{Some of the questions} \textbf{are} rather difficult
                to answer.
                \newline
                \newline
                There \textbf{are} \underline{plenty of opportunities} for young
                men and women.
                \newline
                \newline
                \underline{Half of the eggs} \textbf{were} cracked.
                \newline
                \newline
                \underline{Most of the guests} \textbf{have} left.
        \end{enumerate}
    \item
        \begin{enumerate}
            \item
                \fbox{
                    \begin{tabular}{lll}
                        The number of & + Plural noun & + A singular verb
                    \end{tabular}
                }
                \newline
                \newline
                \underline{The number of cars} in Hong Kong \underline{is} enormous.
                \newline
                \newline
                \underline{The number of boys} in this class \underline{is} twenty.
            \item
                \fbox{
                    \begin{tabular}{lll}
                        A number of & + Plural noun & + A plural verb
                    \end{tabular}
                }
                \newline
                \newline
                \underline{A number of cars} \underline{are} waiting to enter
                the park.
                \newline
                \newline
                \underline{A number of boys} \underline{are} playing in the
                garden.
        \end{enumerate}
    \item
        {\it
        Percentages
        }
        \newline
        \newline
        Percentages and ``the majority" follow the same rule as ``a lot of".
        They take a singular verb when they refer to a singular noun, and have a
        plural verb when they refer to a plural noun.
        \newline
        \newline
        \underline{Ninety percent of this rice} \textbf{is} bad.
        \newline
        \newline
        \underline{Ninety percent of these huts} \textbf{are} illegal.
    \item
        {\it
        ``None" can be used with either a singular verb or a plural verb:
        }
        \newline
        \newline
        \underline{None of this mail} \textbf{is} for me.
        \newline
        \newline
        \underline{None of us} \textbf{are} interested in it.
    \item
        {\it
        ``Other" can be used with singular or plural nouns and verbs.
        \newline
        ``Another" is used only with singular nouns and singular verbs:
        }
        \newline
        \newline
        \underline{Other candidates} \textbf{were} called for their interviews
        this afternoon.
        \newline
        \newline
        There \textbf{isn't} any \underline{other person} suitable for the job.
        \newline
        \newline
        \underline{Another float} \textbf{has} been added to the procession.
        \newline
        \newline
        \textbf{Is} there \underline{another way} of solving the problem?
    %\item
    %    {\it
    %    ``There is" and ``There are"
    %    }
    \item
        {\it
        The verb after ``Who", ``That", ``Which" agrees with noun to which the
        relative pronoun refer.
        }
        \newline
        \newline
        She is one of \underline{the cleverest girls} \textbf{that have}
        graduated from school.
        \newline
        \newline
        She is \underline{the only one} of my friends \textbf{that has} joined
        the police force.
    \item
        {\it
        When the subject is qualified by a phrase or a clause, care should be
        taken that the verb agrees in person and number with its subject, and
        not with the noun or pronoun in the phrase or clause.
        }
        \newline
        \newline
        The \underline{price} of these jeans \textbf{is} reasonable.
        \newline
        \newline
        The \underline{books} borrowed from the library \textbf{are} on my desk.
\end{enumerate}
