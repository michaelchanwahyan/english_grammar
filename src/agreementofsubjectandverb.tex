\section{Agreement of Subject and Verb}

\subsection{General Statement}

\begin{enumerate}
    \item
        {\it
        Subject and verb agree in English
        }
        \newline
        \newline
        \begin{tabular}{llll}
            I am tired. & I was cold. & He has the keys. & Does he look ill? \\
            He is tired. & He was cold. & I have the keys. & Do I look ill? \\
            We are tired. & We were cold. & We have the keys. & Do they look ill?
        \end{tabular}
    \item
        {\it
        Demonstrative adjectives agree with the noun they qualify.
        }
        \newline
        \newline
        \begin{tabular}{ll}
            This book & That boy \\
            These books & Those boys
        \end{tabular}
    \item
        {\it
        The verb after a possessive pronoun depends on what the pronoun refers to.
        }
        \newline
        \newline
        \begin{tabular}{ll}
            This is your shirt. & \underline{Mine IS} over there. (=my shirt is) \\
            These are your shoes. & \underline{Mine ARE} over there. (=my shoes are)
        \end{tabular}
    \item
        {\it
        Verbs like DID in the past tense and also CAN, WOULD, MAY, MIGHT, MUST.
        }
        \newline
        \newline
        \fbox{
            \begin{tabular}{ll}
                You/She/We & \underline{must} hurry.
            \end{tabular}
        }
\end{enumerate}

\subsection{Singular Verbs}
\begin{enumerate}
    \item
        {\it
        ANYBODY, ANYONE, EVERYONE, NOBODY, NO ONE, SOMEBODY, SOMEONE
        }
        \newline
        \newline
        \underline{No one wants} a coca cola.
        \newline
        \underline{Does anybody/anyone want} a coca cola?
        \newline
        \newline
        Mary, \underline{someone/somebody is waiting} for you outside.
        \newline
        \newline
        \underline{No one} knows where the destination is.
    \item
        \fbox{
            \begin{tabular}{llll}
                \underline{One of} & \underline{Every one of} & \underline{Each of} & \multirow{2}{*}{+ plural noun} \\
                \underline{Either of} & \underline{Neither of} & \underline{The number of} &
            \end{tabular}
        }
        \newline
        \newline
        \underline{One of the shops is} still open.
        \newline
        \newline
        \underline{Every one of us is} going to blame him.
        \newline
        \newline
        \underline{Each of the boys has sung} a song to the teacher.
        \newline
        \newline
        \underline{Either of the gentlemen is} good enough to be our teacher.
        \newline
        \newline
        \underline{The number of buses} in Hong Kong \underline{is} great.
        \newline
        \newline
        \textbf{He is \underline{one of {\it those}}
        $\underbrace{\text{who \underline{{\it were}} sentenced to death.}}_
        {\text{noun clause describing ``those (people)"}}$}
    \item
        \fbox{
            \begin{tabular}{ll}
                Every/Each/Many a & + singular noun
            \end{tabular}
        }
        \newline
        \newline
        \underline{Many a man has} come across ups and downs in life.
        \newline
        \newline
        \underline{Each prisoner is provided} with three meals a day.
        \newline
        \newline
        \textbf{Note: A singular verb is still used when 2 nouns qualified by
        ``each" or ``every" even though they are joined by ``and"}
        \newline
        \newline
        \underline{Every boy and every girl was} at play.
        \newline
        \newline
        \underline{Each man and each woman is talking}.
    \item
        {\it
        A plural noun in fact denotes a single thing.
        }
        \begin{enumerate}
            \item {\it Name of a book, a country, a house, a hotel, etc.}
                \newline
                \newline
                \underline{``Gulliver's Travels"} \textbf{is} still widely read.
                \newline
                \newline
                \underline{``Romeo and Juliet"} \textbf{is} one of the works of William Shakespeare.
                \newline
                \newline
                \underline{The United States} \textbf{is} a highly industrialized country.
            \item {\it When distances, weights, heights, time, amount of
                money, etc. represent a single figure of quantity.}
                \newline
                \newline
                \underline{Ten miles} \textbf{is} quite a long way to walk.
                \newline
                \newline
                \underline{Three pounds of grapes} \textbf{costs} fiften dollars.
                \newline
                \newline
                \underline{Six feet} \textbf{is} too high for me to jump over.
                \newline
                \newline
                \underline{Five years in adversity} \textbf{was} not easy to endure.
                \newline
                \newline
                \underline{Two times two} \textbf{is} equal to four.
            \item {\it When 2 singular nouns joined by ``and" suggests one
                single idea or refers to the same person / thing.}
                \newline
                \newline
                \underline{Bread and butter} \textbf{is} my favourite food.
                \newline
                \underline{Bread and butter} \textbf{are} the necessaries of
                life.
                \newline
                \newline
                {\it However when these 2 nouns refer to two persons/things,
                each noun should be preceded by an article respectively. Plural
                verb is used in such a case.}
                \newline
                \newline
                \underline{A poet and novelist} \textbf{has} arrived our school.
                \newline
                \underline{A poet and a novelist} \textbf{have} arrived our
                school.
                \newline
                \newline
                \underline{A white and black dog} \textbf{is} lying by the gate.
                \newline
                \underline{A whie and a black dog} \textbf{are} lying by the
                gate.
                \newline
                \newline
                \underline{My son and student} \textbf{does} well in the examination.
                \newline
                \underline{My son and my student} \textbf{do} well in the
                examination.
            \item {\it Nouns that are plural in form, but singular in meaning.}
                \newline
                \begin{tabular}{lllll}
                    apparatus & billiards & crisis & crossroads & measles \\
                    numps & news & summons & taps & whereabouts
                \end{tabular}
                %\newline
                %\newline
                %No news is good news.
                \newline
                \newline
                \underline{His whereabouts} is still unknown.
        \end{enumerate}
    \item
        {\it
        The subject is an uncountable noun.
        }
        \newline
        \newline
        \underline{Rice} \textbf{is} the daily diet of the Chinese.
        \newline
        \newline
        \underline{Milk} \textbf{is} a highly perishable food.
    \item
        {\it ``Class" nouns like `clothing', `food', `furniture', `crockery',
        `cutlery', `luggage', `footwear', `traffic', `stationary', etc.
        }
\end{enumerate}
