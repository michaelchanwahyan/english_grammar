\section{Phrase}

\subsection{The Meaning of a Phrase}
\begin{enumerate}
    \item a group of related words
    \item containing no finite verbs
    \item forming parts of a sentence
\end{enumerate}

\begin{tabular}{ll}
    Adverb Phrase 
    & Joe is sitting \underline{near the window}. \\
    & \underline{Having finished the work}, David slept. (gerund as DP)\\
    & \underline{In case of fire}, dial 999. \\
    & \underline{In my opinion}, examination is boring. \\
    & \underline{Not knowing her}, I kept quiet. (gerund as DP) \\
    & \underline{Surrounded by many small hills}, the lake is beautiful.
    (participle as DP)\\
    & \underline{In consequence of the heavy rain}, we cancel the meeting. \\
    & I work hard \underline{in order to pass the exam}. \\ \\
    Noun Phrase
    & \underline{The tall boy} is coming. \\
    & \underline{To solve this problem} is easy. (infinitive as NP) \\
    & I dislike \underline{reading long novels}. (gerund as NP) \\
    & Jordan is \underline{a smart fireman}. \\
    & \underline{To learn English} is difficult. (infinitive as NP) \\
    & \underline{Learning Chinese} is also difficult. (gerund as NP) \\
    & \underline{Lending him money} is useless. (gerund as NP) \\
    & Nancy is \underline{a woman of ability} (a woman: NP; of ability: AP) \\
    & Martin, \underline{an intelligent student}, is a nice boy.
\end{tabular}

\subsection{Kinds of Phrases}

\begin{enumerate}
    \item By Function
        \begin{enumerate}
            \item
                {\it
                Noun Phrase
                }
                - a phrase which does the work of noun in a sentence. It can be
                the subject, object, complement of apposition.
                \begin{enumerate}
                    \item \underline{To swim at mid-night} is strange. (subject)
                    \item I know \underline{how to do it}. (object)
                    \item John is \underline{a good student}. (complement)
                    \item Dickson, \underline{my best friend}, was here last
                        night. (apposition)
                \end{enumerate}
            \item
                {\it
                Adjective Phrase
                }
                - a phrase which does the work of an adjective in a sentence.
                \begin{enumerate}
                    \item He is a man \underline{of bravery}. (= He is a brave
                        man.)
                    \item The white building \underline{on the hill} is a
                        castle.
                    \item The body \underline{sitting behind you} is my brother.
                    \item The giro \underline{with long hair} is my sister.
                    \item The shirt is \underline{of high quality}. (= The shirt
                        has high quality.)
                    \item Your effort is \underline{of no use}. (= Your effort
                        is not useful / is useless.)
                    \item Do your homework \underline{with care}. (= Do your
                        homework carefully.)
                \end{enumerate}
            \item
                {\it
                Adverb Phrase
                }
                - a phrase which does the work of an adverb in a sentence.
                \begin{enumerate}
                    \item They returned \underline{in safety}. (= They returned
                        safely.)
                    \item Let's go \underline{at once}. (= Let's go now.)
                    \item \underline{After that accident}, I have fear for great
                        height. (Time)
                    \item He stayed \underline{in the country} last summer.
                        (Place)
                    \item She sang \underline{in a magnetic voice}. (Manner)
                    \item \underline{Because of the heavy rain}, the picnic was
                        cancelled. (Reason)
                    \item \underline{In order to save fuel}, speed limit is
                        reduced to 30 km per hour. (Purpose)
                    \item \underline{Despite the bad weather}, they set off in a
                        raft. (Concession) (=Although the weather is bad, they
                        set off in a raft.)
                    \item She was too sky to \underline{voice her opinion}
                        (Result)
                    \item \underline{In case of non-delivery}, return to the
                        sender. (Condition)
                \end{enumerate}
        \end{enumerate}
    \item By Form
        \begin{enumerate}
            \item
                {\it
                Prepositinoal Phrase
                }
                - consisting of a preposition and its object, it can act as a
                nounm an adjective and an adverb.
                \begin{enumerate}
                    \item She appeared \underline{from behind the door}. (DP)
                    \item \underline{Between six and seven} will suit me. (NP)
                    \item The apples \underline{in the shop window} looked
                        bigger. (AP)
                    \item It is a matter \underline{of the utmost importance}.
                        (AP)
                    \item People are singing \underline{on the bus}. (DP)
                    \item The teacher spoke to the students \underline{in a
                        fatherly manner}. (DP)
                    \item \underline{Despite many failure}, he is not giving up.
                        (DP)
                \end{enumerate}
            \item
                {\it
                Infinitive Phrase
                }
                - consisting of a ``to-infinitive" and other words; it can act
                as noun, an adjective, an adverb and independently.
                \begin{enumerate}
                    \item \underline{To be able to take pain} is one of the
                        conditions of success. (NP)
                    \item The worst thing that you can do is \underline{to copy
                        a letter someone else have written}. (NP)
                    \item He must learn \underline{to work hard}. (NP)
                    \item You need more proofs \underline{to support your
                        theory}. (DP)
                    \item He is not a man \underline{to tell a lie}. (AP)
                    \item We should eat \underline{to live}, not live
                        \underline{to eat}. (DP)
                    \item You are too young \underline{to understand such a
                        complicated matter}. (DP)
                    \item \underline{To tell you the truth}, I don't understand
                        a single word. (independently)
                    \item \underline{To be frank with you}, I dislike him.
                        (independently)
                \end{enumerate}
            \item
                {\it
                Gerund Phrase
                }
                - consistenting of a gerund and its accompanying words, ALWAYS
                used as a noun.
                \begin{enumerate}
                    \item \underline{Writing with the left hand} is more
                        dicficult. (NP)
                    \item She suggested \underline{eating out}. (NP)
                    \item \underline{No smoking}. (NP)
                    \item Her favourite recreation, \underline{playing the
                        piano}, disturbs her neighbours. (NP)
                \end{enumerate}
            \item
                {\it
                Participle Phrase
                }
                - consisting of a participle and other words, ALWAYS used as an
                adjective or adverb.
                \begin{enumerate}
                    \item The person \underline{smoking over there} is my uncle.
                        (AP)
                    \item The village was built on a hill \underline{overlooking
                        the sea}. (AP)
                    \item He brouht some flowers \underline{picked from the
                        garden}. (AP)
                    \item \underline{Having finished his homework}, the little
                        boy watched the television. (AP)
                    \item The dead and \underline{the dying} are lying around
                        him. (``the dying" as AP (the + adjective) referring to
                        people of the same kind, plural verb is used)
                \end{enumerate}
        \end{enumerate}
    \item Word $\leftrightarrow$ Phrase
        \begin{enumerate}
            \item \underline{Adjective $\leftrightarrow$ Adjective Phrase}
                \newline
                \newline
                \fbox{
                    \begin{tabular}{lll}
                        Adjective
                        & $\leftrightarrow$ &
                        Adjective Phrase \\ \hline \hline
                        My mother is a \underline{healthy} woman.
                        & $\leftrightarrow$ &
                        My mother is a woman \underline{of good health}.
                        \\
                        David is an honourable man.
                        & $\leftrightarrow$ &
                        David is a man \underline{of honour}.
                        \\
                        I have never seen a \underline{white} elephant before.
                        & $\leftrightarrow$ &
                        I have never seen an elephant \underline{with a white
                        skin} before.
                        \\
                        I like to see a \underline{smiling} face.
                        & $\leftrightarrow$ &
                        I like to see a face \underline{with a smile on it}.
                        \\
                        Johnson is a \underline{rich} man.
                        & $\leftrightarrow$ &
                        Johnson is a man \underline{with plenty of money}.
                        \\
                        He lives in a deserted village.
                        & $\leftrightarrow$ &
                        He lives in a village \underline{with any inhabitants}.
                    \end{tabular}
                }
            \item \underline{Adverb $\leftrightarrow$ Adverb Phrase}
                \newline
                \newline
                \fbox{
                    \begin{tabular}{lll}
                        Adverb
                        & $\leftrightarrow$ &
                        Adverb Phrase \\ \hline \hline
                        Emily spoke \underline{loudly}
                        & $\leftrightarrow$ &
                        Emily spoke \underline{in a loud voice}.
                        \\
                        Tom ran \underline{quickly}.
                        & $\leftrightarrow$ &
                        Tom ran \underline{at great speed}.
                        \\
                        She replied to my question \underline{rudely}.
                        & $\leftrightarrow$ &
                        She replied to my question \underline{in a rude manner}.
                        \\
                        John is coming \underline{now}.
                        & $\leftrightarrow$ &
                        John is coming \underline{at this moment}.
                        \\
                        No such disease was known \underline{then}.
                        & $\leftrightarrow$ &
                        No suck disease was known \underline{in those days}.
                    \end{tabular}
                }
                \newline
                \newline

                \fbox{
                    \begin{tabular}{lll}
                        Adverb
                        & $\leftrightarrow$ &
                        Adverb Phrase \\ \hline \hline
                        bravely
                        & $\leftrightarrow$ &
                        in a brave manner; with bravery
                        \\
                        wisely
                        & $\leftrightarrow$ &
                        in a wise manner; with wisdome
                        \\
                        swiftly
                        & $\leftrightarrow$ &
                        in a swift mannerm; with swiftness
                        \\
                        beautifully
                        & $\leftrightarrow$ &
                        in a beautiful style; with beauty
                        \\
                        terribly
                        & $\leftrightarrow$ &
                        in a terrible manner
                        \\
                        promptly
                        & $\leftrightarrow$ &
                        in a prompt way; with great promptitude
                        \\
                        imprudently
                        & $\leftrightarrow$ &
                        in an imprudent way; with imprudence
                        \\
                        immediately
                        & $\leftrightarrow$ &
                        at once; without delay
                        \\
                        soon
                        & $\leftrightarrow$ &
                        in the near future;
                        \\
                        now
                        & $\leftrightarrow$ &
                        at this moment; at present
                        \\
                        here
                        & $\leftrightarrow$ &
                        in this place; on this side
                        \\
                        there
                        & $\leftrightarrow$ &
                        at that place
                        \\
                        recently
                        & $\leftrightarrow$ &
                        just now
                    \end{tabular}
                }
                \newline
                \newline
                Not every adverb phrase can be replaced by an adverb.
                \newline
                Examples: He stayed \underline{in the country} last summer.
        \end{enumerate}
    \item Function of a Phrase
        \newline
        \newline
        fire + burns
        \newline
        \newline
        a fierce fire, breaking out yesterday, + burnt down his house
        \newline
        \newline
        a fierce fire, breaking out suddenly yesterday afternoon, +
        completely burnt down his house and many others in the same street
        \newline
        \newline
        a fierce fire, breaking out suddenly yesterday afternoon at 4
        o'clock, + completely burnt down his house and all the others houses
        in the same street except five.
\end{enumerate}
