%\chapter{Present Tense}
\section{Simple Present Tense}
The Simple Present does not refer to present time only.
It can be used to show activities done in past and future time as well.

We use the Simple Present Tense:
\begin{enumerate}
    \item to express present action
        \begin{enumerate}
            \item \underline{for a universal truth.}
                \newline
                The earth rotates on its own axis.
            \item \underline{for a fact.}
                \newline
                Mr. Chan speaks English well.
            \item \underline{for a habitual action.}
                \newline
                She is sometimes lates.
                \newline
                We go to the cinema once in a while.
                \newline
                \textit{Note: in this sense, the tense is used with such adverbs
                or adverb phrases as ALWAYS, USUALLY, GENERALLY, OFTEN,
                FREQUENTLY, SOMETIMES, SELDOM, RARELY, EVERY MONTH, ONCE A WEEK,
                TWICE A MONTH, NOW AND THEN, FROM TIME TO TIME, ONCE IN A WHILE,
                etc.}
            \item \underline{to show an order or request.}
                \newline
                Don't talk.
                \newline
                Please keep quiet.
        \end{enumerate}
    \item to express planned future action
        \begin{enumerate}
            \item \underline{when referring to the future in clauses of time.}
                \newline
                I will tell him when he comes tomorrow.
                \newline
                Wrong: \st{I will tell him when he will come tomorrow.}
                \newline
                Don't go out before Mummy returns.
                \newline
                Wait until Uncle arrives tomorrow afternoon.
            \item \underline{when referring to a future action ACCORDING TO THE 
                TIME-TABLE.}
                \newline
                Verbs that can be used in this way are therefore restricted to
                those connected with travel (come, go, arrive, leave, return,
                depart, sail, etc.) and few others (start, begin, etc.).
                \newline
                Her plane arrives at 2:35 this afternoon.
        \end{enumerate}
\end{enumerate}
