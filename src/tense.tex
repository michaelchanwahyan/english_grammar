\section{Simple Present Tense}
The Simple Present does not refer to present time only.
It can be used to show activities done in past and future time as well.

Uses of the Simple Present tense:
\begin{enumerate}
    \item to express present action
        \begin{enumerate}
            \item \underline{for a universal truth.}
                \newline
                The earth rotates on its own axis.
            \item \underline{for a fact.}
                \newline
                Mr. Chan speaks English well.
            \item \underline{for a habitual action.}
                \newline
                She is sometimes lates.
                \newline
                We go to the cinema once in a while.
                \newline
                \textit{Note: in this sense, the tense is used with such adverbs
                or adverb phrases as ALWAYS, USUALLY, GENERALLY, OFTEN,
                FREQUENTLY, SOMETIMES, SELDOM, RARELY, EVERY MONTH, ONCE A WEEK,
                TWICE A MONTH, NOW AND THEN, FROM TIME TO TIME, ONCE IN A WHILE,
                etc.}
            \item \underline{to show an order or request.}
                \newline
                Don't talk.
                \newline
                Please keep quiet.
        \end{enumerate}
    \item to express planned future action
        \begin{enumerate}
            \item \underline{when referring to the future in clauses of time.}
                \newline
                I will tell him when he comes tomorrow.
                \newline
                Wrong: \st{I will tell him when he will come tomorrow.}
                \newline
                Don't go out before Mummy returns.
                \newline
                Wait until Uncle arrives tomorrow afternoon.
            \item \underline{when referring to a future action ACCORDING TO THE 
                TIME-TABLE.}
                \newline
                Verbs that can be used in this way are therefore restricted to
                those connected with travel (come, go, arrive, leave, return,
                depart, sail, etc.) and few others (start, begin, etc.).
                \newline
                Her plane arrives at 2:35 this afternoon.
        \end{enumerate}
\end{enumerate}

\section{Present Continuous Tense}
The main use of this tense is to express TEMPORARY action.
For two verbs, it sometimes does not matter whether we use this tense or the
Simple Present tense.
All the following sentences are correct:
\begin{center}
\begin{tabular}{|cl|}
    \hline
    \multirow{2}{*}{Live} & a) Mary lives at Tai Po now. \\
                          & b) Mary is living at Tai Po now. \\ 
                          & \\
    \multirow{2}{*}{Work} & c) David works in a factory now. \\
                          & d) David is working in a factory now. \\ \hline
\end{tabular}
\end{center}

However, for most other verbs, we use the Simple Present tense for a habit,
and the Present Continuous tense for temporary action, e.g.
\begin{center}
\begin{tabular}{|cl|}
    \hline
    Habit     & e) It usually rains in the summer months. \\
    Temporary & f) It is raining now. \\
              & \\
    Habit     & g) Dogs often chase cats. \\
    Temporary & h) Our dog is sleeping now. \\ \hline
\end{tabular}
\end{center}

Uses of the Present Continuous tense
\begin{enumerate}
    \item \underline{to express temporary action (which may stop soon)}
        \begin{enumerate}
            \item Mary is hiding rom us. Where is she?
            \item Is Anne helping Mummy in the kitchen?
        \end{enumerate}
    \item \underline{to express planned future action} \newline
        We cannot use this tense with all verbs.
        It is used with common verbs such as:
        \begin{center}
        \begin{tabular}{lllll}
            go & meet & leave & take part in & visit \\
            & & & & \\
            come & see & arrive & depart & get
        \end{tabular}
        \end{center}
        \begin{enumerate}
            \item Daddy is getting a new car tomorrow. \newline
                = ~ Daddy will get a new car tomorrow.
            \item Mary is taking part in a chess tournament next week. \newline
                = Mary will take part in the chess tournament next week.
        \end{enumerate}
    \item \underline{to express an action which is repeated over a period of time}
        \begin{enumerate}
            \item Mr. Chan is marking the papers these few days. \newline
            \item Continuously showing (mainly negative) feeling. \newline
                \begin{tabular}{cccl}
                    \multirow{3}{*}{Be}
                    & Constantly
                    & \multirow{3}{*}{+ing}
                    & \underline{This is used to indicate a frequently repeated action,}\\
                    & Always       & & \underline{and usually suggest continuaully that the speaker is} \\
                    & Continuously & & \underline{irritated by the action or think that it is unreasonable.}
                \end{tabular}
            \item They are always arguing about money or something else.
            \item Why are you always complaining about the food?
        \end{enumerate}
    \item \underline{we do not use Present Continuous form of some verbs.} \newline
        \underline{We have to use their Simple Present form instead. They include:}
        \begin{enumerate}
            \item verbs of emotion or of the senses: \newline
                \begin{tabular}{llllll}
                    want & desire & hate & like & hear & smell \\
                    wish & forgive & adore & dislike & see & notice
                \end{tabular}
            \item verbs of thinking or possessing: \newline
                \begin{tabular}{llllll}
                    know & mean & believe & forget & understand & think (when expressing an opinion)\\
                    have & own & owe & belong & possess & remember
                \end{tabular} \newline
                However, when some of the verbs listed above are used in a
                particular sense, we may use the continuous tense:
                \begin{enumerate}
                    \item see = meet; interview; consult (a doctor, etc.); visit \newline
                        I am seeing my lawyer this afternoon.
                    \item hear = receive news \newline
                        I have been hearing a lot about him lately.
                    \item have = eat; drink; take; engage in; enjoy \newline
                        Miss Lam is having a lesson at the moment.
                \end{enumerate}
        \end{enumerate}
\end{enumerate}

\section{Present Perfect Tense}
The main use of this tense is to express past action when we do not know or say
the time of the action.
In many cases, the action has a connection with the present (often through the
results of the action) but the NOT necessarily the case.
Compare these correct sentences: \newline
\begin{tabular}{ll}
    \it{time stated:} & Peter whent to Macau last week. \\ & \\
    \it{time not stated:} & Peter has been to Macau. \\ & \\
    \it{time stated:} & I saw this film about three months ago. \\ & \\
    \it{time not stated:} & I have seen this film before.
\end{tabular}
