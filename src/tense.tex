\onecolumn
\section{Overview of Tense}
\begin{table}[h]
    \centering
    \begin{tabular}{|l|c|c|c|}
        \hline
        \multicolumn{4}{|c|}{Active Voice} \\ \hline
        & Present Tense & Past Tense & Future Tense \\ \hline
        Simple & teach & taught & will teach \\ \hline
        Continuous & is teaching & was teaching & will be teaching \\ \hline
        Perfect & has taught & had taught & will have taught \\ \hline
        Perfect Continuous & has been teaching & had been teaching & will have been teaching \\ \hline \hline
        \multicolumn{4}{|c|}{Passive Voice} \\ \hline
        & Present Tense & Past Tense & Future Tense \\ \hline
        Simple & is taught & was taught & will be taught \\ \hline
        Continuous & is being taught & was being taught & $\times$ \\ \hline
        Perfect & has been taught & had been taught & will have been taught \\ \hline
        Perfect Continuous & $\times$ & $\times$ & $\times$ \\ \hline
    \end{tabular}
    \caption{Tense}
\end{table}

\section{Tense: Present}
\subsection{Simple Present Tense}
The Simple Present does not refer to present time only.
It can be used to show activities done in past and future time as well.

Uses of the Simple Present tense:
\begin{enumerate}
    \item to express present action
        \begin{enumerate}
            \item \underline{for a universal truth.}
                \newline
                The earth rotates on its own axis.
            \item \underline{for a fact.}
                \newline
                Mr. Chan speaks English well.
            \item \underline{for a habitual action.}
                \newline
                She is sometimes lates.
                \newline
                We go to the cinema once in a while.
                \newline
                \textit{Note: in this sense, the tense is used with such adverbs
                or adverb phrases as ALWAYS, USUALLY, GENERALLY, OFTEN,
                FREQUENTLY, SOMETIMES, SELDOM, RARELY, EVERY MONTH, ONCE A WEEK,
                TWICE A MONTH, NOW AND THEN, FROM TIME TO TIME, ONCE IN A WHILE,
                etc.}
            \item \underline{to show an order or request.}
                \newline
                Don't talk.
                \newline
                Please keep quiet.
        \end{enumerate}
    \item to express planned future action
        \begin{enumerate}
            \item \underline{when referring to the future in clauses of time.}
                \newline
                I will tell him when he comes tomorrow.
                \newline
                Wrong: \st{I will tell him when he will come tomorrow.}
                \newline
                Don't go out before Mummy returns.
                \newline
                Wait until Uncle arrives tomorrow afternoon.
            \item \underline{when referring to a future action ACCORDING TO THE 
                TIME-TABLE.}
                \newline
                Verbs that can be used in this way are therefore restricted to
                those connected with travel (come, go, arrive, leave, return,
                depart, sail, etc.) and few others (start, begin, etc.).
                \newline
                Her plane arrives at 2:35 this afternoon.
        \end{enumerate}
\end{enumerate}

\subsection{Present Continuous Tense}
The main use of this tense is to express TEMPORARY action.
For two verbs, it sometimes does not matter whether we use this tense or the
Simple Present tense.
All the following sentences are correct:
\begin{center}
\begin{tabular}{|cl|}
    \hline
    \multirow{2}{*}{Live} & a) Mary lives at Tai Po now. \\
                          & b) Mary is living at Tai Po now. \\ 
                          & \\
    \multirow{2}{*}{Work} & c) David works in a factory now. \\
                          & d) David is working in a factory now. \\ \hline
\end{tabular}
\end{center}

However, for most other verbs, we use the Simple Present tense for a habit,
and the Present Continuous tense for temporary action, e.g.
\begin{center}
\begin{tabular}{|cl|}
    \hline
    Habit     & e) It usually rains in the summer months. \\
    Temporary & f) It is raining now. \\
              & \\
    Habit     & g) Dogs often chase cats. \\
    Temporary & h) Our dog is sleeping now. \\ \hline
\end{tabular}
\end{center}

Uses of the Present Continuous tense
\begin{enumerate}
    \item \underline{to express temporary action (which may stop soon)}
        \begin{enumerate}
            \item Mary is hiding rom us. Where is she?
            \item Is Anne helping Mummy in the kitchen?
        \end{enumerate}
    \item \underline{to express planned future action}
        \newline
        We cannot use this tense with all verbs.
        It is used with common verbs such as:
        \begin{center}
        \begin{tabular}{lllll}
            go & meet & leave & take part in & visit \\
            & & & & \\
            come & see & arrive & depart & get
        \end{tabular}
        \end{center}
        \begin{enumerate}
            \item Daddy is getting a new car tomorrow.
                \newline
                = ~ Daddy will get a new car tomorrow.
            \item Mary is taking part in a chess tournament next week.
                \newline
                = Mary will take part in the chess tournament next week.
        \end{enumerate}
    \item \underline{to express an action which is repeated over a period of time}
        \begin{enumerate}
            \item Mr. Chan is marking the papers these few days.
                \newline
            \item Continuously showing (mainly negative) feeling.
                \newline
                \begin{tabular}{cccl}
                    \multirow{3}{*}{Be}
                    & Constantly
                    & \multirow{3}{*}{$+$ ~ ing}
                    & \underline{This is used to indicate a frequently repeated action,}\\
                    & Always       & & \underline{and usually suggest continuaully that the speaker is} \\
                    & Continuously & & \underline{irritated by the action or think that it is unreasonable.}
                \end{tabular}
            \item They are always arguing about money or something else.
            \item Why are you always complaining about the food?
        \end{enumerate}
    \item \underline{we do not use Present Continuous form of some verbs.}
        \newline
        \underline{We have to use their Simple Present form instead. They include:}
        \begin{enumerate}
            \item verbs of emotion or of the senses:
                \newline
                \begin{tabular}{llllll}
                    want & desire & hate & like & hear & smell \\
                    wish & forgive & adore & dislike & see & notice
                \end{tabular}
                \newline
            \item verbs of thinking or possessing:
                \newline
                \begin{tabular}{llllll}
                    know & mean & believe & forget & understand & think (when expressing an opinion)\\
                    have & own & owe & belong & possess & remember
                \end{tabular}
                \newline
                However, when some of the verbs listed above are used in a
                particular sense, we may use the continuous tense:
                \newline
                \begin{enumerate}
                    \item see = meet; interview; consult (a doctor, etc.); visit
                        \newline
                        I am seeing my lawyer this afternoon.
                        \newline
                    \item hear = receive news
                        \newline
                        I have been hearing a lot about him lately.
                        \newline
                    \item have = eat; drink; take; engage in; enjoy
                        \newline
                        Miss Lam is having a lesson at the moment.
                        \newline
                \end{enumerate}
        \end{enumerate}
\end{enumerate}

\subsection{Present Perfect Tense}
The main use of this tense is to express past action when we do not know or say
the time of the action.
In many cases, the action has a connection with the present (often through the
results of the action) but the NOT necessarily the case.
Compare these correct sentences:
\newline
\begin{tabular}{ll}
    \it{time stated:} & Peter whent to Macau last week. \\ & \\
    \it{time not stated:} & Peter has been to Macau. \\ & \\
    \it{time stated:} & I saw this film about three months ago. \\ & \\
    \it{time not stated:} & I have seen this film before.
\end{tabular}
\newline
\newline
The present perfect tense is used:
\begin{enumerate}
    \item \underline{with ``for" + a period of time or ``since" + a point of time}
        \newline
        \newline
        \begin{tabular}{ll}
            for & It has rained for at least an hour. \\
                & I have waited for you for more than twenty minutes. \\
            since & Mr. Wong has not been here since early March. \\
                  & I have lived in Hong Kong ever since I was born.
        \end{tabular}
    \item \underline{a past action with a result which can still be observed or felt in the present}
        \newline
        \newline
        I can't inform her of the date of the meeting.
        I have lost her phone number.
        \newline
        \newline
        Note: the adverbs usually used with the tense in this sense include:
        \newline
        \textbf{EVER, NEVER, BEFORE, YET, RECENTLY, LATELY, SO FAR, UP TO NOW,
        UP TO THE PRESENT}.
        \newline
        \newline
        She doesn't know that man. She has never seen him before.
        \newline
        \newline
        He can't come with us. He hasn't finished his homework yet.
        \newline
        \newline
        This tense is never used with adverbs of past time such as:
        \newline
        \textbf{YESTERDAY, LAST WEEK, TWO DAYS AGO, etc.}
        The Simple Past tense should be used instead.
        \newline
        \newline
        I met Mr. Lam a few days ago.
    \item \underline{for an immediate past action using \textbf{JUST}}
        \newline
        \newline
        \begin{tabular}{ll}
            voice on the phone: & May I speak to Mr. Chan please? \\
            secretary: & Mr. Chan has just left.
        \end{tabular}
    \item \underline{with ``yet" in questions and negtive statements}
        \newline
        \newline
        Have you put up the curtains yet?
        \newline
        \newline
        I haven't had a shower yet.
        \newline
        \newline
        Here ``yet" means ``up ot the moment of speaking".
    \item \underline{for an action which was repeated several times in the past}
        \newline
        \newline
        Please note the adverbs listed below:
        \newline
        \newline
        \begin{tabular}{|l|l|l|}
            \hline
            subj. + verb (trans.) & adverb / adverb phrase & phrase \\
            \hline
            \multirow{7}{*}{I have warned him}
            & more than once & not to do that. \\
            & twice & \\
            & at least three times & \\
            & several times & \\
            & many times & \\
            & over an over again & \\
            & on several occasions & \\ \hline
        \end{tabular}
        \newline
        \newline
        The following points should also be remembered:
        \begin{enumerate}
            \item the verb in the time clauses introduced by \textbf{SINCE} is
                always in the past tense.
                \newline
                \newline
                They have known each other since they were children.
                \newline
            \item we also use the present perfect tense after
                \textbf{IT / THIS IS THE FIRST TIME}.
                \newline
                \newline
                This is the first time the pop group has come to Hong Kong.
                \newline
                \newline
        \end{enumerate}
\end{enumerate}

\subsection{Present Perfect Continuous Tense}
This tense is used:
\begin{enumerate}
    \item \underline{for an action which started in the past,}
        \newline
        \underline{is still going on at present,}
        \newline
        \underline{and will probably go into the future.}
        \newline
        \newline
        \begin{tabular}{ll}
            \multirow{2}{*}{He has been watching television}
            & since 7 p.m. \\
            & for at least three hours.
        \end{tabular}
        \newline
        \newline
        Compare the following and you will see the present perfect continuous
        tense is different from the present perfect tense.
        \newline
        \newline
        ``Mummy, may I take a rest now?" asked little Tommy to his mother,
        ``I have read fourteen pages of that awful book."
    \item \underline{for immediate past non-stop action}
        \newline
        \newline
        Where have you been? ~~~ I have been looking for you all morning.
        \newline
        \newline
        Note: to give emphasis to the continuity of the action, the adverbs used
        often have \textbf{ALL} before them:
        \newline
        \newline
        ALL this morning, ALL this month, ALL this week, etc.
        If however, the number of times the action repeated is mentioned,
        the present perfect tense should be used instead.
        \newline
        \newline
        I have been telephoning you all morning but each time the line was engaged.
        \newline
        \newline
        I have telephoned you at least four times this morning but each time the line was engaged.
\end{enumerate}

\section{Tense: Past}
\subsection{Simple Past Tense}
The Past simple tense is used to show an action which was completed at a
definite point of time in the past.
It is usually used with adverbs indicating a certain time in the past like
YESTERDAY, THE DAY BEFORE YESTERDAY, LAST WEEK, ONCE, THEN, IN THE PAST, ONCE
UPON A TIME, LONG, LONG AGO, THREE YEARS AGO, IN THOSE DAYS, IN 1976, JUST NOW,
WHEN I WAS A CHILE, etc.

She fell down and broke her wrist yesterday.
\newline
Note:
\begin{enumerate}
    \item \underline{To express an action which took place in the past, we may
        use either the present}
        \newline
        \underline{perfect tense or the Simple Past tense. In meaning, we use
        the present perfect tense}
        \newline
        \underline{when we wish to show THE PRESENT result of the past action.}
        \newline
        \newline
        Visitor: May I see the manager, please?
        \newline
        \newline
        Secretary: Sorry, he has just left.
        \newline
        \newline
        Here the secretary uses the present perfect tense because her intention
        is to tell the visitor that the manager IS NOT AVAILABLE at the moment.
        \newline
        \newline
        In constructino, the present perfect tense cannot be used with an adverb
        of past time.
        When the time of the action occurred is mentioned, therefore, the Simple
        Past tense should be used.
        \newline
        \newline
        I have lost my indentity card.
        \newline
        \newline
        I lost my identity card yesterday.
        \newline
        \newline
        In addition, the Simple Past tense is usually used when there is an
        adverb of place in the sentence.
        \newline
        \newline
        I have bought a mobile phone.
        \newline
        \newline
        I bought a mobile phone at Broadway.
        \newline
        \newline
        Simple Past tense is also used when the sentence is a question about
        time.
        \newline
        \newline
        When did you arrive?
        \newline
        \newline
        Wrong: When have you arrived?
    \item \underline{Compare the examples below. Note that we use different
        tenses with different adverbs}
        \newline
        \underline{or when the same adverb is used with different meanings.}
        \newline
        \newline
        He was once a policeman. (once = formerly)
        \newline
        \newline
        He has visited the Space Museum once. (once = for one time)
        \newline
        \newline
        She had just finished her homework. (just = not long ago)
        \newline
        \newline
        She was here just now. (just now = a short time ago)
    \item \underline{The adverbs ALWAYS, OFTEN, etc are not always used with
        simple present tense.}
        \newline
        \underline{The simple past tense when used to show a repeated or
        habitual action in the past,}
        \underline{are also accompanied by these adverbs.}
        \newline
        \newline
        Wrong: \st{A few years ago I lived in Shatin. I always get up early
        because I have to travel a long way to work.}
        \newline
        \newline
        Right: A few years ago, I lived in Shatin. I always got up early because
        I had to travel a long way to work.
        \newline
        \newline
        \underline{A past habit can also be expressed by using WOULD or USED TO}
        \newline
        I spent my childhood in the New Territories. Those were really happy
        days. Everyday I would / used to go to explore new places with my
        classmates after school.
    \item \underline{We use the Simple Past tense in the following construction
        when we are refferring to}
        \newline
        \underline{present unreal situations.}
        \begin{enumerate}
            \item
                \begin{tabular}{lll}
                    \multirow{3}{*}{It is} & time       & \multirow{3}{*}{we
                    told her the truth / she were told the truth.} \\
                                           & about time & \\
                                           & high time  &
                \end{tabular}
                \newline
                Compare: It is time for us to tell her the truth
            \item
                \begin{tabular}{ll}
                    \multirow{2}{*}{I wish} & he were here now. \\
                    & I knew the answer now.
                \end{tabular}
            \item I would rather you did it now. / tomorrow.
        \end{enumerate}
        Note that the preference here concerns another person who is not the
        subject.
        \newline
        Compare: I would rather do it now.
\end{enumerate}

\subsection{Past Continuous Tense}
This tense is used for:
\begin{enumerate}
    \item \underline{an action continuing at a definite past time or when
        another action took place.}
        \newline
        They were having a meeting at that time.
        As I was watching television, the telephone rang.
        \newline
        In this sense, the tense is often found with adverbs like JUST THEN, AT
        THAT TIME, AT THAT MOMENT, WHEN THE FIRE BROKE OUT, etc.
    \item \underline{an action that was in progress over a certain period in the
        past}
        \newline
        They were talking for the whole of the class.
        \newline
        I was helping my brother with his lessons from 7 to 9 yesterday evening.
        \newline
        I was raining all last night.
    \item \underline{an action that was frequently repeated in the past with a
        touch of the speaker's feeling.}
        \newline
        I found her presence unberable. She was always / constantly /
        continuously complaining about something or the other.
\end{enumerate}

\subsection{Past Perfect Tense}
The Past Perfect tense is used {\bf\it{ONLY}} when we wish to emphasize that an
action was completed before another past action or before a point of time in the
past.
\newline
\newline
When she arrived at the station, the train had already left.
\newline
By the end of last year, he had already written two books.
\newline
Wrong: \st{He had come back five minutes ago.}
\newline
Right: He came back five minutes ago.
\newline
\newline
This tense is generally found
\begin{enumerate}
    \item \underline{in reported speech}
        \newline
        \newline
        ``I have returned the book to the library," he said.
        \newline
        He said that he had returned the book to the library.
        \newline
        ``I saw Jane yesterday," Bill said.
        \newline
        Bill said that he had seen Jane the day before.
        \newline
        {\bf Note that the Present Perfect and the Simple Past tense in direct
        speech usually become the Past Perfect Tense in reported speech.}
    \item \underline{in complex sentences}
        \newline
        \newline
        When I woke up, it had already stopped raining.
        \newline
        He suddenly remembered that he had not locked the door.
    \item \underline{in sentences where there are such phrases as UP TO THAT
        TIME, BY THAT TIME,}
        \newline
        \underline{BY THE END OF LAST MONTH, BY 2017 etc.}
        \newline
        \newline
        By the end of 2017, the company had constructed two flyovers.
    \item \underline{after IT WAS / THAT WAS THE FIRST TIME}
        \newline
        \newline
        It was the first time he had come to Hong Kong.
\end{enumerate}

\subsection{Past Perfect Continuous Tense}
The Past Perfect Continuous tense is used to express the duration of an action
up to a point of past time or when another action tok place.
\newline
By the end of last month, Mr. Chan had been teaching in this school for ten
years.
\newline
When Tom arrived, the teacher had been teaching for an hour.
\newline
\newline
Compare the examples below and you will see how we use the Past Perfect tense
and Past Perfect Continuous tense in different situations:
\newline
\newline
She did not want to read that book because she had read it before.
\newline
She did not want to stop although she had been reading for three hours.

\newpage
\section{Tense: Future}
\subsection{Simple Future Tense}
The simple future tense is used for an action which will happen sometime after
now. It is usually found with such adverbs as SOON, SHORTLY, TOMORROW, THE DAY
AFTER TOMORROW, IN A FEW MINUTES, LATER, SOME DAY, NEXT WEEK, etc.
\newline
\newline
Note:
\begin{enumerate}
    \item \underline{Traditionally, {\bf SHALL} is used with the first person
        and {\bf WILL} with the second and}
        \newline
        \underline{third persons. However there is a tendency today to use
        {\bf WILL} for all persons. In spite}
        \newline
        \underline{of this, the first person
        interrogative is almost always {\bf SHALL I / SHALL WE}?}
        \newline
        \newline
        Shall I turn on the fan? (=Would you like me to turn on the fan?)
    \item \underline{Future actions can also be expressed by using:}
        \newline
        \newline
        \begin{enumerate}
            \item \underline{be going to}
                \newline
                Mr. Lee is going to redecorate his flat this year.
                \newline
                \newline
                {\it This construction is used to show:}
                \begin{enumerate}
                    \item \underline{intention}
                        \newline
                        I am going to see him this evening.
                    \item \underline{probability}
                        \newline
                        I think / I am afraid it is going to rain.
                    \item \underline{certainty}
                        \newline
                        Hurry up, Mary. You are going to miss the last ferry.
                \end{enumerate}
                {\it This construction is {\bf NOT} used:}
                \begin{enumerate}
                    \item \underline{for pure futurity}
                        \newline
                        Wrong: \st{I am going to be seventeen next year.}
                        \newline
                        Right: I shall / will be seventeen next year.
                    \item \underline{in conditional sentences}
                        \newline
                        Wrong: \st{he is going to fail if he does not work
                        hard.}
                        \newline
                        Right: He will fail if he does not work hard.
                    \item \underline{the Present Continuous tense}
                        \newline
                        We are holding a party next Saturday. Would you like to
                        join us?
                    \item \underline{the Simple Present tense}
                        \newline
                        My train leaves at 6:35 tomorrow morning.
                \end{enumerate}
        \end{enumerate}
\end{enumerate}

\subsection{Future Continuous Tense}
\underline{This tense is used for an action which will be still going on at some
time in the future.}
\newline
\newline
This time tomorrow I shall be travelling on the train for Guangzhou.
\newline
When you come to Hong Kong again next month, our children will be taking the
HKDSE.

\subsection{Future Perfect Tense}
The Future Perfec tense is used for an action which will be completed before a
point of time in the future or before the occurrence of another future action.
It is usually used in a complex sentenced or is accompanied by such adverbs as
BY TOMORROW, BY THEN, BY THE END OF ..., BY THE TIME ..., etc.
\newline
newline
By the end of his year, we shall have completed the training course.
\newline
Mr. Wong will have taught in this school for 25 years when he retires next onth.

\subsection{Future Perfect Continuous Tense}
\underline{The Future Perfect Continuous tense is used to emphasize the
continuity of an actino up}
\newline
\underline{to a future time or after it.}
\newline
\newline
When this programme finishes, you will have been watching television for three
whole hours.

\subsection{Should \& Would}
\begin{enumerate}
    \item \underline{As the past tense of SHALL and WILL, SHOULD and WOULD are
        used to show}
        \newline
        \underline{future in the past.}
        \newline
        \newline
        He said, ``I will come again tomorrow."
        \newline
        He said that he would come again the next day.
    \item \underline{SHOULD is also used with all persons expressing:}
        \begin{enumerate}
            \item obligation = ought to
                \newline
                \newline
                \begin{tabular}{lll}
                    \multirow{2}{*}{You} & ourhgt to & \multirow{2}{*}{do it now.} \\
                    & should &
                \end{tabular}
                \newline
                Note: To show past unfulfilled obligation, we use SHOULD / OUGHT
                TO + HAVE + PAST PASTICIPLE:
                \newline
                \newline
                \begin{tabular}{lll}
                    \multirow{2}{*}{You} & ourhgt to & \multirow{2}{*}{have done it yesterday.
                    Why didn't you do it?} \\
                    & should &
                \end{tabular}
            \item probability or expectation
                \newline
                \newline
                He should be at home now, I think.
                \newline
                I've given him full directions. He should have no difficulty in
                finding the place.
                \newline
                \newline
                Note: To refer to an incomplete activity, we use the perfect
                infinitive after SHOULD:
                \newline
                \newline
                They should have arrived at the airport by now.
            \item a polite remark
                \newline
                \newline
                You are mistaken. I should say.
                \newline
                Compare: You are mistaken.
                \newline
                \newline
                Marry: I'm putting on weight, aren't I?
                \newline
                Bill: Yes, I should think so.
                \newline
                \newline
                I should be most grateful if you would kindly consider my
                application and grant me an interview at your earlier
                convenience.
                \newline
                \newline
                Note: the pattern I SHOULD BE MOST / VERY GRATEFUL IF YOU WOULD
                / COULD is very common in business letters.
            \item surprise, annoyance or indignation
                \newline
                \newline
                X: The telephone is ringing. Answer it.
                \newline
                Y: Why should I?
                \newline
                \newline
                X: Where is my dictionary?
                \newline
                Y: How should I know?
        \end{enumerate}
    \item \underline{WOULD is also used:}
        \begin{enumerate}
            \item for polite requests
                \begin{enumerate}
                    \item WOULD YOU ... ?
                        \newline
                        \newline
                        Would you sign your name here, pleace?
                        \newline
                        Compare: Will you sign your name here, pleace?
                        \newline
                        note: WOUDL YOU is considered more polite.
                    \item WOULD YOU MIND + GERUND / WOULD  YOU MIND IF ...
                        \newline
                        \newline
                        Would you mind lending me your dictionary?
                        \newline
                        Would you mind if I borrow your dictionary?
                \end{enumerate}
            \item to show preference
                \newline
                \newline
                I would rather stay at home and watch television than go to the
                cinema.
                \newline
                \newline
                {\it Note: WOULD RATHER is followed by a ``bare" infinitive. But
                when the preference concerns another person who is not the
                subject of the verb, we use either the simple past tense or the
                past perfect tense depending on whether we are referring to
                present or past unreal situation.}
                \newline
                \newline
                I would rather you came tomorrow.
                \newline
                I would rather you had come yesterday.
                \newline
                \newline
                See also how WOULD RATHER differ from PREFER in usgae:
                \newline
                \newline
                I would rather walk than wait for another tram.
                \newline
                I prefer walking to waiting for another tram.
            \item for past habitual or repeated actions
                \newline
                \newline
                When he playecd in the school team, he would practise every
                Saturday.
        \end{enumerate}
\end{enumerate}
