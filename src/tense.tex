\section{Overview of Tense}
\begin{table}
    \centering
    \begin{tabular}{|l|c|c|c|}
        \hline
        \multicolumn{4}{|c|}{Active Voice} \\ \hline
        & Present Tense & Past Tense & Future Tense \\ \hline
        Simple & teach & taught & will teach \\ \hline
        Continuous & is teaching & was teaching & will be teaching \\ \hline
        Perfect & has taught & had taught & will have taught \\ \hline
        Perfect Continuous & has been teaching & had been teaching & will have been teaching \\ \hline \hline
        \multicolumn{4}{|c|}{Passive Voice} \\ \hline
        & Present Tense & Past Tense & Future Tense \\ \hline
        Simple & is taught & was taught & will be taught \\ \hline
        Continuous & is being taught & was being taught & will be taught \\ \hline
        Perfect & has been taught & had been taught & will have been taught \\ \hline
    \end{tabular}
    \caption{Tense}
\end{table}

\newpage
\section{Simple Present Tense}
The Simple Present does not refer to present time only.
It can be used to show activities done in past and future time as well.

Uses of the Simple Present tense:
\begin{enumerate}
    \item to express present action
        \begin{enumerate}
            \item \underline{for a universal truth.}
                \newline
                The earth rotates on its own axis.
            \item \underline{for a fact.}
                \newline
                Mr. Chan speaks English well.
            \item \underline{for a habitual action.}
                \newline
                She is sometimes lates.
                \newline
                We go to the cinema once in a while.
                \newline
                \textit{Note: in this sense, the tense is used with such adverbs
                or adverb phrases as ALWAYS, USUALLY, GENERALLY, OFTEN,
                FREQUENTLY, SOMETIMES, SELDOM, RARELY, EVERY MONTH, ONCE A WEEK,
                TWICE A MONTH, NOW AND THEN, FROM TIME TO TIME, ONCE IN A WHILE,
                etc.}
            \item \underline{to show an order or request.}
                \newline
                Don't talk.
                \newline
                Please keep quiet.
        \end{enumerate}
    \item to express planned future action
        \begin{enumerate}
            \item \underline{when referring to the future in clauses of time.}
                \newline
                I will tell him when he comes tomorrow.
                \newline
                Wrong: \st{I will tell him when he will come tomorrow.}
                \newline
                Don't go out before Mummy returns.
                \newline
                Wait until Uncle arrives tomorrow afternoon.
            \item \underline{when referring to a future action ACCORDING TO THE 
                TIME-TABLE.}
                \newline
                Verbs that can be used in this way are therefore restricted to
                those connected with travel (come, go, arrive, leave, return,
                depart, sail, etc.) and few others (start, begin, etc.).
                \newline
                Her plane arrives at 2:35 this afternoon.
        \end{enumerate}
\end{enumerate}

\newpage
\section{Present Continuous Tense}
The main use of this tense is to express TEMPORARY action.
For two verbs, it sometimes does not matter whether we use this tense or the
Simple Present tense.
All the following sentences are correct:
\begin{center}
\begin{tabular}{|cl|}
    \hline
    \multirow{2}{*}{Live} & a) Mary lives at Tai Po now. \\
                          & b) Mary is living at Tai Po now. \\ 
                          & \\
    \multirow{2}{*}{Work} & c) David works in a factory now. \\
                          & d) David is working in a factory now. \\ \hline
\end{tabular}
\end{center}

However, for most other verbs, we use the Simple Present tense for a habit,
and the Present Continuous tense for temporary action, e.g.
\begin{center}
\begin{tabular}{|cl|}
    \hline
    Habit     & e) It usually rains in the summer months. \\
    Temporary & f) It is raining now. \\
              & \\
    Habit     & g) Dogs often chase cats. \\
    Temporary & h) Our dog is sleeping now. \\ \hline
\end{tabular}
\end{center}

Uses of the Present Continuous tense
\begin{enumerate}
    \item \underline{to express temporary action (which may stop soon)}
        \begin{enumerate}
            \item Mary is hiding rom us. Where is she?
            \item Is Anne helping Mummy in the kitchen?
        \end{enumerate}
    \item \underline{to express planned future action}
        \newline
        We cannot use this tense with all verbs.
        It is used with common verbs such as:
        \begin{center}
        \begin{tabular}{lllll}
            go & meet & leave & take part in & visit \\
            & & & & \\
            come & see & arrive & depart & get
        \end{tabular}
        \end{center}
        \begin{enumerate}
            \item Daddy is getting a new car tomorrow.
                \newline
                = ~ Daddy will get a new car tomorrow.
            \item Mary is taking part in a chess tournament next week.
                \newline
                = Mary will take part in the chess tournament next week.
        \end{enumerate}
    \item \underline{to express an action which is repeated over a period of time}
        \begin{enumerate}
            \item Mr. Chan is marking the papers these few days.
                \newline
            \item Continuously showing (mainly negative) feeling.
                \newline
                \begin{tabular}{cccl}
                    \multirow{3}{*}{Be}
                    & Constantly
                    & \multirow{3}{*}{$+$ ~ ing}
                    & \underline{This is used to indicate a frequently repeated action,}\\
                    & Always       & & \underline{and usually suggest continuaully that the speaker is} \\
                    & Continuously & & \underline{irritated by the action or think that it is unreasonable.}
                \end{tabular}
            \item They are always arguing about money or something else.
            \item Why are you always complaining about the food?
        \end{enumerate}
    \item \underline{we do not use Present Continuous form of some verbs.}
        \newline
        \underline{We have to use their Simple Present form instead. They include:}
        \begin{enumerate}
            \item verbs of emotion or of the senses:
                \newline
                \begin{tabular}{llllll}
                    want & desire & hate & like & hear & smell \\
                    wish & forgive & adore & dislike & see & notice
                \end{tabular}
                \newline
            \item verbs of thinking or possessing:
                \newline
                \begin{tabular}{llllll}
                    know & mean & believe & forget & understand & think (when expressing an opinion)\\
                    have & own & owe & belong & possess & remember
                \end{tabular}
                \newline
                However, when some of the verbs listed above are used in a
                particular sense, we may use the continuous tense:
                \newline
                \begin{enumerate}
                    \item see = meet; interview; consult (a doctor, etc.); visit
                        \newline
                        I am seeing my lawyer this afternoon.
                        \newline
                    \item hear = receive news
                        \newline
                        I have been hearing a lot about him lately.
                        \newline
                    \item have = eat; drink; take; engage in; enjoy
                        \newline
                        Miss Lam is having a lesson at the moment.
                        \newline
                \end{enumerate}
        \end{enumerate}
\end{enumerate}

\newpage
\section{Present Perfect Tense}
The main use of this tense is to express past action when we do not know or say
the time of the action.
In many cases, the action has a connection with the present (often through the
results of the action) but the NOT necessarily the case.
Compare these correct sentences:
\newline
\begin{tabular}{ll}
    \it{time stated:} & Peter whent to Macau last week. \\ & \\
    \it{time not stated:} & Peter has been to Macau. \\ & \\
    \it{time stated:} & I saw this film about three months ago. \\ & \\
    \it{time not stated:} & I have seen this film before.
\end{tabular}
\newline
\newline
The present perfect tense is used:
\begin{enumerate}
    \item \underline{with ``for" + a period of time or ``since" + a point of time}
        \newline
        \newline
        \begin{tabular}{ll}
            for & It has rained for at least an hour. \\
                & I have waited for you for more than twenty minutes. \\
            since & Mr. Wong has not been here since early March. \\
                  & I have lived in Hong Kong ever since I was born.
        \end{tabular}
    \item \underline{a past action with a result which can still be observed or felt in the present}
        \newline
        \newline
        I can't inform her of the date of the meeting.
        I have lost her phone number.
        \newline
        \newline
        Note: the adverbs usually used with the tense in this sense include:
        \newline
        \textbf{EVER, NEVER, BEFORE, YET, RECENTLY, LATELY, SO FAR, UP TO NOW,
        UP TO THE PRESENT}.
        \newline
        \newline
        She doesn't know that man. She has never seen him before.
        \newline
        \newline
        He can't come with us. He hasn't finished his homework yet.
        \newline
        \newline
        This tense is never used with adverbs of past time such as:
        \newline
        \textbf{YESTERDAY, LAST WEEK, TWO DAYS AGO, etc.}
        The Simple Past tense should be used instead.
        \newline
        \newline
        I met Mr. Lam a few days ago.
    \item \underline{for an immediate past action using \textbf{JUST}}
        \newline
        \newline
        \begin{tabular}{ll}
            voice on the phone: & May I speak to Mr. Chan please? \\
            secretary: & Mr. Chan has just left.
        \end{tabular}
    \item \underline{with ``yet" in questions and negtive statements}
        \newline
        \newline
        Have you put up the curtains yet?
        \newline
        \newline
        I haven't had a shower yet.
        \newline
        \newline
        Here ``yet" means ``up ot the moment of speaking".
    \item \underline{for an action which was repeated several times in the past}
        \newline
        \newline
        Please note the adverbs listed below:
        \newline
        \newline
        \begin{tabular}{|l|l|l|}
            \hline
            subj. + verb (trans.) & adverb / adverb phrase & phrase \\
            \hline
            \multirow{7}{*}{I have warned him}
            & more than once & not to do that. \\
            & twice & \\
            & at least three times & \\
            & several times & \\
            & many times & \\
            & over an over again & \\
            & on several occasions & \\ \hline
        \end{tabular}
        \newline
        \newline
        The following points should also be remembered:
        \begin{enumerate}
            \item the verb in the time clauses introduced by \textbf{SINCE} is
                always in the past tense.
                \newline
                \newline
                They have known each other since they were children.
                \newline
            \item we also use the present perfect tense after
                \textbf{IT / THIS IS THE FIRST TIME}.
                \newline
                \newline
                This is the first time the pop group has come to Hong Kong.
                \newline
                \newline
        \end{enumerate}
\end{enumerate}

\newpage
\section{Present Perfect Continuous Tense}
This tense is used:
\begin{enumerate}
    \item \underline{for an action which started in the past,}
        \newline
        \underline{is still going on at present,}
        \newline
        \underline{and will probably go into the future.}
        \newline
        \newline
        \begin{tabular}{ll}
            \multirow{2}{*}{He has been watching television}
            & since 7 p.m. \\
            & for at least three hours.
        \end{tabular}
        \newline
        \newline
        Compare the following and you will see the present perfect continuous
        tense is different from the present perfect tense.
        \newline
        \newline
        "Mummy, may I take a rest now?" asked little Tommy to his mother,
        "I have read fourteen pages of that awful book."
    \item \underline{for immediate past non-stop action}
        \newline
        \newline
        Where have you been? ~~~ I have been looking for you all morning.
        \newline
        \newline
        Note: to give emphasis to the continuity of the action, the adverbs used
        often have \textbf{ALL} before them:
        \newline
        \newline
        ALL this morning, ALL this month, ALL this week, etc.
        If however, the number of times the action repeated is mentioned,
        the present perfect tense should be used instead.
        \newline
        \newline
        I have been telephoning you all morning but each time the line was engaged.
        \newline
        \newline
        I have telephoned you at least four times this morning but each time the line was engaged.
\end{enumerate}
