\section{Active and Passive Voice}

\begin{enumerate}
    \item
        {\it
        When the subject is the doer of the action, the verb is said to be
        active.
        }
        \newline
        John \underline{wrote} the poem.
        \newline
        \newline
        {\it
        When the subject is the reeeivers or sufferer of the action, the verb is
        said to be passive.
        }
        \newline
        The poem \underline{was written} by John.
        \newline
        \newline
        {\it
        The most important use of the passive voice is for an action where the
        doer of he action is unknown or is unimportant.
        }
        \newline
        The rich businessman \underline{was found} lying dead in his hotel room.
        \newline
        \newline
        Flyovers \underline{have been built} in many places to ease traffic
        congestion.
    \item
        {\it
        The passive voice is formed by using the appropriate tense of the verb
        ``to be" + the past participle of the verb.
        }
        \newline
        The following table shows the passive voice of the different tenses:
        \newline
        \begin{table}[h]
            \centering
            \begin{tabular}{|l|l|l|}
                \hline
                Tense
                & Verb Form
                & Examples \\ \hline \hline
                Simple Present
                & am/are/is + p.p.
                & Mr. Wong teaches this class. \\
                & & This class is taught by Mr. Wong.\\ \hline
                Simple Past
                & was/were + p.p.
                & Mr. Wong taught this class. \\
                & & This class was taught by Mr. Wong. \\ \hline
                Simple Future
                & shall/will + be + p.p.
                & Mr. Wong will teach this class. \\
                & & This class will be taught by Mr. Wong. \\ \hline \hline
                Present Continuous
                & am/are/is + being + p.p.
                & Mr. Wong is teaching this class. \\
                & & This class is being taught by Mr. Wong. \\ \hline
                Past Continuous
                & was/were + being + p.p.
                & Mr. Wong was teaching this class. \\
                & & This class was being taught by Mr. Wong. \\ \hline \hline
                Present Perfect
                & have/has + been + p.p.
                & Mr. Wong has taught this class. \\
                & & This class has been taught be Mr. Wong \\ \hline
                Past Perfect
                & had + been + p.p.
                & Mr. Wong had taught this class. \\
                & & This class had been taught by Mr. Wong. \\ \hline
                Future Perfect
                & shall/will + have + been + p.p.
                & Mr. Wong will have taught this class. \\
                & & This class will have been taught by Mr. Wong. \\ \hline
            \end{tabular}
            \caption{Note: The tense that are not included in this table do not
            have the passive voice.}
        \end{table}
    \item
        {\it
        If the verb contains anyone of CAN, MAY, MUST, HOULD, OUGHT TO, HAVE TO, NEED (NOT),
        USED TO, BE GOING TO, BE TO, the passive voice is formed by using the
        same verb + be + the past participle of the verbs.
        }
        \newline
        \newline
        \begin{tabular}{rcl}
            \multirow{3}{*}{He}
            & can/must/may/should &
            \multirow{3}{*}{answer the question in this way.} \\
            & has to/ought to/need not & \\
            & is going to/is to/used to & \\ \\
            \multirow{3}{*}{The question}
            & can/must/may/should &
            \multirow{3}{*}{be answered in this way (by him).} \\
            & has to/ought to/need not & \\
            & is going to/is to/used to & \\ \\
        \end{tabular}
    \item
        {\it
        Since the subject in the passive sentence is derived from the object in
        the active one, only transitive verbs can be used in the passive voice.
        This is important because:
        }
        \begin{enumerate}
            \item in changing a sentence from active into passive voice, you can
                ignore the intransitive verb.
                \newline
                \newline
                I \underline{am} sure she \underline{will invite} us to the party.
                \newline
                I \underline{am} sure we \underline{will be invited} to the party (by her)
            \item in writing compositions, you can avoid making mistakes like
                these:
                \newline
                \newline
                Wrong: \st{The accident was happened/was occurred yesterday.}
                \newline
                Right: The accident happened / occurred yesterday.
                \newline
                \newline
                Wrong: \st{The driver was died instantly.}
                \newline
                Right: The driver died instantly. / The driver was killed
                instantly.
        \end{enumerate}
    \item
        {\it
        In changing active voice into passive voice, the doer of the action (by
        + agent) is usually omitted when:
        }
        \begin{enumerate}
            \item \underline{it is a general item}
                \newline
                \newline
                The farmers grow these vegetables in the New Territories.
                \newline
                These vegetables are grown in the New Territories.
                \newline
                \newline
                Compare: My uncle grows these vegetables in the New Territories.
                \newline
                These vegetables are grown by my uncle in the New Territories.
            \item \underline{it is unknown}
                \newline
                \newline
                Someone has kinapped the rich man's son.
                \newline
                The rich man's son has been kinapped.
                \newline
                \newline
                Compare: Tom has posted the letter.
                \newline
                The letter has been posted by Tom.
            \item \underline{it is a pronoun}
                \newline
                \newline
                He is distributing the papers.
                \newline
                The papers are being distributed.
                \newline
                \newline
                Comapre: Mary is laying the table.
                \newline
                The table is being laid by Mary.
        \end{enumerate}
    \item
        {\it
        Some verbs are followed by prepositions. When these verbs are changed
        into the passive voice, the prepositions keep their position after the
        verb.
        }
        \newline
        \newline
        I don't think they will turn down your application.
        \newline
        I don't think your application will be turned down.
    \item
        {\it
        The negative of the passive voice is formed by adding the word NOT after
        the auxiliary verb (am, are, is, shall, will, etc.) and the
        interrogative by putting the auxiliary verb before the subject.
        }
        \newline
        \newline
        Affirmative: The photograph \underline{was taken} by my brother.
        \newline
        Negative: The photograph \underline{was not taken} by my brother.
        \newline
        Interrogative: \underline{Was} the photograph \underline{taken} by your brother?
        \newline
        \newline
        {\it
        With this in mind, you will not find it difficult to turn the following
        sentences into the passive voice:
        }
        \newline
        \newline
        \begin{tabular}{ll}
            We will not hold the meeting tomorrow.
            & The meeting \underline{will not be held} tomorrow. \\
            She did not clean the windows yesterday.
            & The windows \underline{were not cleaned} yesterday. \\
            Do people speak English in many countries?
            & \underline{Is} English \underline{spoken} in many countries? \\
            How are you going to solve the problem?
            & How \underline{is} the problem \underline{going to be solved}? \\
            Where did you put the key?
            & Where \underline{was} the key \underline{put}?
        \end{tabular}
    \item
        {\it
        Sentences beginning with PEOPLE SAY, PEOPLE BELIEVE, PEOPLE KNOW, etc.
        may be changed into the passive voice in either of these two ways:
        }
        \newline
        \newline
        People say that he is an ambitious young man.
        \newline
        \underline{It is said} that he is an ambitious young man.
        \newline
        \newline
        They say that she was a famous singer.
        \newline
        \underline{It is said} that she was a famous singer.
    \item
        {\it
        Imperatives can be changed into the passive voice in these ways:
        }
        \newline
        \begin{tabular}{ll}
            Turn off the light.
            & You \underline{are told} to turn off the light.
            \\
            Put this waste paper in the litter bin!
            & This waste paper \underline{should be put} in the litter bin!
        \end{tabular}
    \item
        {\it
        Some berbs are usually used in the active form though they may imply a
        passive meaning. The most common ones are:
        }
        \newline
        \newline
        {\bf cut, hang, measure, look, smeel, write, wash, open, sound, pay,
        feel, taste, read, sell, ring}
        \newline
        \newline
        Right: The door-bell rang.
        \newline
        Wrong: \st{The door-bell was rung.}
        \newline
        \newline
        Right: The job pays well.
        \newline
        Wrong: \st{The job is paid well.}
\end{enumerate}
